\documentclass[a4paper,10pt]{article}
\usepackage[utf8]{inputenc}

%opening
\title{}
\author{}

\begin{document}

\maketitle

\begin{abstract}

\end{abstract}

\section{Scanning protocol of the robot}
Everytime the robot performs a scan, it turns on itself and measures its distance to the surrounding walls. 
It then writes onto a string representing pixels the surrounding area it scanned (the pixels are either walls, free space, or unknown if it is outside scanning radius or behind an obstacle).
It alternates the measures on the sonar and the gyroscope, and everytime it hits a wall, the pixel that sits at that location is turned into a wall.


\subsection{Add the free space pixels}
Cercles consécutifs avec des pas de pixel_size/2

listes de starts et de stop 

\subsection{Refining the method}
Instead of directly asigning the value of wall to a pixel during the scanning, a short buffer 

dist = mean(buffer)
pixel_angle_width = pixel_width/dist
pixel

Size of the buffer:
The faster the robot turns, the less values it wil be able to record, and therefore, the sorter the buffer should be.
Roughly:
Nb_recorded_measures = recording_freq*360/turning_speed.
Thus, buffer_size \linear to \alpha/turning_speed.

\end{document}
